\documentclass[11pt,a4paper]{article}

\usepackage{pslatex}
\usepackage{inconsolata}
\usepackage[square,numbers]{natbib}
\usepackage[includehead,top=2.5cm,bottom=2.5cm,
            left=3cm,right=3cm]{geometry}
\usepackage{listings}
\usepackage{parskip}
\usepackage{amsmath}
\usepackage{amssymb}

\usepackage{color}
\definecolor{bluish}{rgb}{0.20,0.29,0.46}
% *always \use this last*
\usepackage[colorlinks,breaklinks,pdftex,bookmarks=true,
            linkcolor=bluish,citecolor=bluish,urlcolor=bluish]{hyperref}

\usepackage{xcolor}

\definecolor{codegreen}{rgb}{0,0.6,0}
\definecolor{codegray}{rgb}{0.5,0.5,0.5}
\definecolor{codepurple}{rgb}{0.58,0,0.82}
\definecolor{backcolour}{rgb}{0.95,0.95,0.92}

\lstdefinestyle{mystyle}{
    backgroundcolor=\color{backcolour},
    commentstyle=\color{codegreen},
    numberstyle=\tiny\color{codegray},
    stringstyle=\color{codepurple},
    basicstyle=\ttfamily\footnotesize,
    breakatwhitespace=false,
    breaklines=true,
    captionpos=b,
    keepspaces=true,
    numbers=left,
    numbersep=5pt,
    showspaces=false,
    showstringspaces=false,
    showtabs=false,
    tabsize=2
}

\lstset{style=mystyle}

\usepackage{enumitem}
\setlist{topsep=0ex,itemsep=2pt,partopsep=0pt,parsep=0pt,leftmargin=7.5mm}

% Bold face for vectors
\renewcommand{\vec}[1]{\mathbf{#1}}

% Sublist style for drafting
\renewcommand\labelenumii{\theenumii.}
\renewcommand\theenumii{\arabic{enumii}}
\renewcommand\theenumiii{\arabic{enumiii}}

\begin{document}
\lstset{language=bash}

{\huge User Guide for Spowtd v0.3.0}\\[2ex]
{\large Alex Cobb}\\[0ex]

\renewcommand{\baselinestretch}{1.18}\normalsize

This is the user guide for Spowtd, which implements the scalar
parameterization of water table dynamics described in
\citet{Cobb_et_al_2017} and \citet{Cobb_and_Harvey_2019}.

\section{The steps of scalar parameterization}
Scalar parameterization involves these essential steps:
\begin{enumerate}
\item Load water level, precipitation and evapotranspiration data;
\item Identify dry intervals and storm intervals;
\item Match intervals of rising water levels to rainstorms;
\item Construct a master rising curve;
\item Construct a master recession curve;
\item Fit a preliminary specific yield function to the master rising
  curve;
\item Jointly fit a specific yield and a conductivity (equivalently,
  transmissivity) function to the master rising and recession curves.
\end{enumerate}

\section{The spowtd script}
The \texttt{spowtd} script provides a command-line interface to
perform calculations with Spowtd.

\subsection{Dependencies}
Running the script requires Python 3 and the Python packages
\href{https://matplotlib.org/}{Matplotlib},
\href{https://numpy.org/}{Numpy}, and
\href{https://pypi.org/project/pytz/}{Pytz}.

\subsection{Using the script}
The \texttt{spowtd} script has these subcommands (typically run in
this order):
\begin{itemize}
\item \texttt{spowtd load}: Load water level, precipitation and
  evapotranspiration data.
\item \texttt{spowtd classify}: Classify data into storm and
  interstorm intervals.
\item \texttt{spowtd set-zeta-grid}: Set up water level grid for
  master curves.
\item \texttt{spowtd recession}: Assemble recession curve.
\item \texttt{spowtd rise}: Assemble rise curve.
\item \texttt{spowtd plot}: Plot data.
\item \texttt{spowtd set-curvature}: Set site curvature.
\item \texttt{spowtd simulate}: Simulate data rise curve, recession
  curve, or rising and receding intervals.
\item \texttt{spowtd pestfiles}: Generate input files for calibration
  with PEST.
\end{itemize}

The first step is to load the precipitation, evapotranspiration and
water level data.  The input text files must be in an UTF-8-compatible
encoding (ASCII is fine).  The time zone is stored with the dataset
and will be used in plots (all times are stored internally as UNIX
timestamps).  For example, to load data into a new dataset file called
\texttt{ekolongouma.sqlite3}:
\begin{lstlisting}[frame=single]
spowtd load ekolongouma.sqlite3 \
  -vvv \
  --precipitation src/precipitation_Ekolongouma.txt \
  --evapotranspiration src/evapotranspiration_Ekolongouma.txt \
  --water-level src/waterlevel_Ekolongouma.txt \
  --timezone Africa/Lagos
\end{lstlisting}
The verbosity flags (\texttt{-vvv}) are not required; they cause the
script to report more on what is being done.

Next, classify the water level and precipitation time series into
storm and interstorm intervals based on thresholds for rainfall
intensity and rates of increase in water level.  For example, this
command classifies intervals with precipitation of at least 4~mm / h
as storms, and intervals in which the water level is increasing at a
rate of least 8~mm / h as storm response.
\begin{lstlisting}[frame=single]
spowtd classify ekolongouma.sqlite3 \
  -vvv \
  --storm-rain-threshold-mm-h 4.0 \
  --rising-jump-threshold-mm-h 8.0
\end{lstlisting}

At this stage the classification can be plotted.  A basic interactive
plot showing the classified water level and precipitation time series
can be produced with:
\begin{lstlisting}[frame=single]
spowtd plot time-series ekolongouma.sqlite3
\end{lstlisting}
An additional panel showing evapotranspiration is plotted if the
\texttt{-e} or \texttt{--plot-evapotranspiration} flag is passed.  The
parts of the water level time series marked as interstorms are on a
light red background, and the parts of the water level time series
marked as storm response are on a light green background.  The parts
of the precipitation time series marked as storms are on a light blue
background.  You can pan in the plot with the right mouse button and
zoom with a left mouse button, or use the magnifying glass to zoom in.
You can revert to earlier zoom and pan values with the arrow buttons.

Adding \texttt{-f} or \texttt{--flags} highlights the parts of the
water level time series that have been classified as storm response
and interstorms, and the parts of the precipitation time series
\begin{lstlisting}[frame=single]
spowtd plot time-series ekolongouma.sqlite3  -f
\end{lstlisting}
The rising intervals are highlighted in blue, intervals with rising
intervals that could not be matched to rain storms are highlighted in
magenta, and rain storms are highlighted in red.

The next step is to establish a uniform grid for water levels.  This
grid is used when storm and interstorm intervals are assembled into
rising and recession curves.
\begin{lstlisting}[frame=single]
spowtd set-zeta-grid -vvv ekolongouma.sqlite3
\end{lstlisting}

The next two steps assemble the recession and rise curves:
\begin{lstlisting}[frame=single]
spowtd recession -vvv ekolongouma.sqlite3
\end{lstlisting}

\begin{lstlisting}[frame=single]
spowtd rise -vvv ekolongouma.sqlite3
\end{lstlisting}

The recession and rise curves are now assembled, and can be plotted.
\begin{lstlisting}[frame=single]
spowtd plot recession ekolongouma.sqlite3
\end{lstlisting}

\begin{lstlisting}[frame=single]
spowtd plot rise ekolongouma.sqlite3
\end{lstlisting}
These plots can be interacted with in the same way: left mouse button
to pan, right mouse button to zoom, disk icon to save.

\section{Parameterization}
Parameters are provided to \texttt{spowtd} in
\href{https://yaml.org/}{YAML} format.

Currently two types of parameter sets are supported: (1) Cubic spline
for specific yield, piecewise linear for the logarithm of
conductivity; and (2) The PEATCLSM parameterization.

The spline parameterizations look like this:
\begin{lstlisting}[frame=single]
specific_yield:
  type: spline
  zeta_knots_mm:
    - -291.7
    - -183.1
    - -15.74
    - 10.65
    - 38.78
    - 168.3
  sy_knots:  # Specific yield, dimensionless
    - 0.1358
    - 0.1671
    - 0.2541
    - 0.2907
    - 0.2892
    - 0.6857
transmissivity:
  type: spline
  zeta_knots_mm:
    - -291.7
    - -5.167
    - 168.3
    - 1000
  K_knots_km_d:  # Conductivity, km/d
    - 5.356e-3
    - 1.002
    - 6577.0
    - 8.430e+3
  minimum_transmissivity_m2_d: 7.442  # Minimum transmissivity, m2/d
\end{lstlisting}
and the PEATCLSM parameterizations look like this:
\begin{lstlisting}[frame=single]
specific_yield:
  type: peatclsm
  sd: 0.162  # standard deviation of microtopographic distribution, m
  theta_s: 0.88  # saturated moisture content, m^3/m^3
  b: 7.4  # shape parameter, dimensionless
  psi_s: -0.024  # air entry pressure, m
transmissivity:
  type: peatclsm
  Ksmacz0: 7.3  # m/s
  alpha: 3  # dimensionless
  zeta_max_cm: 5.0
\end{lstlisting}
(the text following each parameter, after the \verb|#|, is a comment
and invisible to spowtd).

The specific yield and transmissivity curves can be plotted with
\begin{lstlisting}[frame=single]
  spowtd plot WHAT parameters.yml
\end{lstlisting}
where \texttt{WHAT} is one of \texttt{specific-yield},
\texttt{conductivity} or \texttt{transmissivity} and
\texttt{parameters.yml} is a YAML file containing hydraulic
parameters.

The plotting commands \texttt{plot rise}, \texttt{plot recession} and
\texttt{plot time-series} support a parameter \texttt{-p},
\texttt{--parameters}; if a YAML file containing hydraulic parameters
is passed to one of these commands, the corresponding plot (rising
curve, recession curve, rising and receding intervals) is simulated
using those parameters.

The simulated curves and corresponding data can be obtained as text
using \texttt{spowtd simulate WHAT data.sqlite3 parameters.yml} where
\texttt{WHAT} is \texttt{rise}, \texttt{recession}, or
\texttt{intervals}.  These commands write simulated data, water level
data, and / or residuals to an output file (standard output by
default) as delimited text.  For example,
\begin{lstlisting}[frame=single]
spowtd simulate rise ekolongouma.sqlite3 parameters.yml
\end{lstlisting}
reads data from \texttt{ekolongouma.sqlite3} and parameters from the
file \texttt{parameters.yml} and writes the assembled and simulated
rise curves to standard output.

To simulate (or plot) recession requires setting the large-scale
curvature of the site.  The command
\begin{lstlisting}[frame=single]
  spowtd set-curvature ekolongouma.sqlite3 1.0
\end{lstlisting}
sets the site curvature to $1~\text{m}/\text{km}/\text{km}$, whereafter
\begin{lstlisting}[frame=single]
  spowtd simulate recession ekolongouma.sqlite3 parameters.yml
\end{lstlisting}
simulates the water table recession.

\section{Calibration with PEST}
The simulation scripts make it possible to calibrate the specific
yield and transmissivity functions against rise and recession of the
water level using the \href{https://pesthomepage.org/}{PEST} software
package and tools for model-independent parameter estimation and
uncertainty analysis.  It should also be possible to calibrate using
\href{https://www.usgs.gov/software/pest-software-suite-parameter-estimation-uncertainty-analysis-management-optimization-and}{PEST++},
which is designed to have the same text-based interface, by following
a similar procedure.

PEST is a highly configurable set of tools.  One of its strengths is
that it is possible to start with a fairly simple approach and
incorporate more sophisticated functionality as it is needed.  As an
introduction, we illustrate calibration of specific yield parameters
against the rise curve.

For a calibration with PEST, you need to create five text files:
\begin{enumerate}
\item A PEST control file (\texttt{.pst}), which configures how PEST
  will perform the calibration (including identifying the other files
  used during calibration);
\item A parameter template file (\texttt{.tpl}), into which PEST will
  substitute parameter values in a format that can be read by Spowtd;
\item An output template file, or PEST ``instruction file''
  (\texttt{.ins}), which teaches PEST how to extract ``observations''
  from \texttt{spowtd simulate} output;
\item A vector of initial parameters (\texttt{.par}) to start the
  calibration; and
\item A script to execute the rise simulation.
\end{enumerate}

The first step is to create the PEST control file (\texttt{.pst})
following the PEST documentation \citep{Doherty_2010}.  For a PEATCLSM
parameterization, the control file will describe the four PEATCLSM
parameters for specific yield (\texttt{sd}, \texttt{theta\_s},
\texttt{b}, and \texttt{psi\_s}), each with their own parameter group.
The control file must also include an ``observation data'' section
with a single line giving mean dynamic storage for each water level in
the rise curve. The ``model command line'' section must provide the
command line needed to run the script to generate the rise curve.  The
script itself can be, for example, a bash script that calls
\texttt{spowtd simulate rise}.  Finally, the ``model input/output''
section specifies the path to the parameter template file, the path to
the the parameter file that PEST will create by substituting parameter
values into the template, the path to the instruction file
(\texttt{.ins}) that PEST uses to interpret the simulation output, and
the path to the output file created by a single run of the simulation
script.

The parameter template file (\texttt{.tpl}) and the parameter vector
file (\texttt{.par}) are created by replacing values in a Spowtd YAML
parameter file by placeholders, as described in the PEST
documentation.  In the case of calibration of PEATCLSM specific yield
parameters against a rise curve, the template file might look like this:
\begin{lstlisting}[frame=single]
ptf @
specific_yield:
  type: peatclsm
  sd: @sd                      @
  theta_s: @theta_s                 @
  b: @b                       @
  psi_s: @psi_s                   @
transmissivity:
  type: peatclsm
  Ksmacz0: 7.3  # m/s
  alpha: 3  # dimensionless
  zeta_max_cm: 5.0
\end{lstlisting}
and an initial parameter vector file might look like this:
\begin{lstlisting}[frame=single]
double point
          sd    0.162                      1.000000         0.000000
     theta_s    0.88                       1.000000         0.000000
           b    7.4                        1.000000         0.000000
       psi_s    -0.024                     1.000000         0.000000
\end{lstlisting}

To verify the format of a template \texttt{rise\_pars.yml.tpl} and
initial parameters \texttt{rise\_init.par}, use the PEST
\texttt{tempchek} command:
\begin{lstlisting}[frame=single]
tempchek rise_pars.yml.tpl rise_pars.yml rise_init.par
\end{lstlisting}
This command should exit without errors and produce a valid parameter
file at \texttt{rise\_pars.yml}.

The parameter file can then be verified by running your script.  Your
script might, for example, contain the command
\begin{lstlisting}[frame=single]
spowtd simulate rise ekolongouma.sqlite3 rise_pars.yml -o rise_observations.yml --observations
\end{lstlisting}
which generates simulated dynamic storage values (without water levels
or measured dynamic storage values) in
\texttt{rise\_observations.yml}; in PEST, simulated output values are
referred to as ``observations.''

The resulting output file can then be checked against a PEST
instruction file (\texttt{.ins}) that you create for extracting
observation data, which might be called
\texttt{rise\_observations.ins}, using the PEST command
\texttt{inschek}:
\begin{lstlisting}[frame=single]
inschek rise_observations.ins rise_observations.yml
\end{lstlisting}

To then ensure that the correct initial parameters are used in the
calibration, substitute these into the control file using
\texttt{parrep}
\begin{lstlisting}[frame=single]
parrep rise_init.par rise_calibration.in.pst rise_calibration.pst
\end{lstlisting}

To then calibrate specific yield parameters against the rise curve
(alone) using the PEST control file \texttt{rise\_calibration}, call:
\begin{lstlisting}[frame=single]
pestchek rise_calibration &&
(pest rise_calibration.pst ;
 tempchek rise_pars.yml.tpl rise_opt.yml rise_calibration.par)
\end{lstlisting}
These commands check the PEST control file, perform the calibration,
and then substitute the calibrated parameter values from
\texttt{rise\_calibration.par} into \texttt{rise\_opt.yml}.

You can then examine the fit by plotting the rise curve with the
calibrated parameters:
\begin{lstlisting}[frame=single]
spowtd plot rise ekolongouma.sqlite3 --parameters rise_opt.yml
\end{lstlisting}

\subsection{Generating PEST input files with Spowtd}
As a convenience, Spowtd can generate input files for calibration with
PEST, either against the rise curve (\texttt{spowtd pestfiles rise})
or against both rise and recession curves (\texttt{spowtd pestfiles
  curves}).  The arguments to both subcommands are the same.  Taking
calibration against the rise curve as an example, a template file
can be created with
\begin{lstlisting}[frame=single]
spowtd pestfiles rise ekolongouma.sqlite3 parameters.yml tpl \\
  -o rise_parameters.yml.tpl
\end{lstlisting}
An instruction file can similarly be created with
\begin{lstlisting}[frame=single]
spowtd pestfiles rise ekolongouma.sqlite3 parameters.yml ins \\
  -o ekolongouma_rise_observations.ins
\end{lstlisting}
and a control file can be created with
\begin{lstlisting}[frame=single]
spowtd pestfiles rise ekolongouma.sqlite3 parameters.yml pst \\
  -o ekolongouma_rise_calibration.in.pst
\end{lstlisting}
The template and instruction files can be used as-is.  The generated
PEST control file will require substitution of valid starting
parameters and bounds, substitution of paths to input files and the
invocation for simualtion, adjustment of PEST control parameters, etc.

\bibliographystyle{unsrtnat}
\bibliography{user_guide.bib}

\end{document}
